\documentclass[journal,12pt,onecolumn]{IEEEtran}
%
\usepackage{setspace}
\usepackage{gensymb}
%\doublespacing
\singlespacing

%\usepackage{graphicx}
%\usepackage{amssymb}
%\usepackage{relsize}
\usepackage[cmex10]{amsmath}
%\usepackage{amsthm}
%\interdisplaylinepenalty=2500
%\savesymbol{iint}
%\usepackage{txfonts}
%\restoresymbol{TXF}{iint}
%\usepackage{wasysym}
\usepackage{amsthm}
%\usepackage{iithtlc}
\usepackage{mathrsfs}
\usepackage{txfonts}
\usepackage{stfloats}
\usepackage{bm}
\usepackage{cite}
\usepackage{cases}
\usepackage{subfig}
%\usepackage{xtab}
\usepackage{longtable}
\usepackage{multirow}
%\usepackage{algorithm}
%\usepackage{algpseudocode}
\usepackage{enumitem}
\usepackage{mathtools}
\usepackage{steinmetz}
\usepackage{tikz}
\usepackage{circuitikz}
\usepackage{verbatim}
\usepackage{tfrupee}
\usepackage[breaklinks=true]{hyperref}
%\usepackage{stmaryrd}
\usepackage{tkz-euclide} % loads  TikZ and tkz-base
%\usetkzobj{all}
\usetikzlibrary{calc,math}
\usepackage{listings}
    \usepackage{color}                                            %%
    \usepackage{array}                                            %%
    \usepackage{longtable}                                        %%
    \usepackage{calc}                                             %%
    \usepackage{multirow}                                         %%
    \usepackage{hhline}                                           %%
    \usepackage{ifthen}                                           %%
  %optionally (for landscape tables embedded in another document): %%
    \usepackage{lscape}     
\usepackage{multicol}
\usepackage{chngcntr}
%\usepackage{enumerate}

%\usepackage{wasysym}
%\newcounter{MYtempeqncnt}
\DeclareMathOperator*{\Res}{Res}
%\renewcommand{\baselinestretch}{2}
\renewcommand\thesection{\arabic{section}}
\renewcommand\thesubsection{\thesection.\arabic{subsection}}
\renewcommand\thesubsubsection{\thesubsection.\arabic{subsubsection}}

\renewcommand\thesectiondis{\arabic{section}}
\renewcommand\thesubsectiondis{\thesectiondis.\arabic{subsection}}
\renewcommand\thesubsubsectiondis{\thesubsectiondis.\arabic{subsubsection}}

% correct bad hyphenation here
\hyphenation{op-tical net-works semi-conduc-tor}
\def\inputGnumericTable{}                                 %%

\lstset{
%language=C,
frame=single, 
breaklines=true,
columns=fullflexible
}
%\lstset{
%language=tex,
%frame=single, 
%breaklines=true
%}

\begin{document}
%


\newtheorem{theorem}{Theorem}[section]
\newtheorem{problem}{Problem}
\newtheorem{proposition}{Proposition}[section]
\newtheorem{lemma}{Lemma}[section]
\newtheorem{corollary}[theorem]{Corollary}
\newtheorem{example}{Example}[section]
\newtheorem{definition}[problem]{Definition}
%\newtheorem{thm}{Theorem}[section] 
%\newtheorem{defn}[thm]{Definition}
%\newtheorem{algorithm}{Algorithm}[section]
%\newtheorem{cor}{Corollary}
\newcommand{\BEQA}{\begin{eqnarray}}
\newcommand{\EEQA}{\end{eqnarray}}
\newcommand{\define}{\stackrel{\triangle}{=}}

\bibliographystyle{IEEEtran}
%\bibliographystyle{ieeetr}


\providecommand{\mbf}{\mathbf}
\providecommand{\pr}[1]{\ensuremath{\Pr\left(#1\right)}}
\providecommand{\qfunc}[1]{\ensuremath{Q\left(#1\right)}}
\providecommand{\sbrak}[1]{\ensuremath{{}\left[#1\right]}}
\providecommand{\lsbrak}[1]{\ensuremath{{}\left[#1\right.}}
\providecommand{\rsbrak}[1]{\ensuremath{{}\left.#1\right]}}
\providecommand{\brak}[1]{\ensuremath{\left(#1\right)}}
\providecommand{\lbrak}[1]{\ensuremath{\left(#1\right.}}
\providecommand{\rbrak}[1]{\ensuremath{\left.#1\right)}}
\providecommand{\cbrak}[1]{\ensuremath{\left\{#1\right\}}}
\providecommand{\lcbrak}[1]{\ensuremath{\left\{#1\right.}}
\providecommand{\rcbrak}[1]{\ensuremath{\left.#1\right\}}}
\theoremstyle{remark}
\newtheorem{rem}{Remark}
\newcommand{\sgn}{\mathop{\mathrm{sgn}}}
\providecommand{\abs}[1]{\left\vert#1\right\vert}
\providecommand{\res}[1]{\Res\displaylimits_{#1}} 
\providecommand{\norm}[1]{\left\lVert#1\right\rVert}
%\providecommand{\norm}[1]{\lVert#1\rVert}
\providecommand{\mtx}[1]{\mathbf{#1}}
\providecommand{\mean}[1]{E\left[ #1 \right]}
\providecommand{\fourier}{\overset{\mathcal{F}}{ \rightleftharpoons}}
%\providecommand{\hilbert}{\overset{\mathcal{H}}{ \rightleftharpoons}}
\providecommand{\system}{\overset{\mathcal{H}}{ \longleftrightarrow}}
	%\newcommand{\solution}[2]{\textbf{Solution:}{#1}}
\newcommand{\solution}{\noindent \textbf{Solution: }}
\newcommand{\cosec}{\,\text{cosec}\,}
\providecommand{\dec}[2]{\ensuremath{\overset{#1}{\underset{#2}{\gtrless}}}}
\newcommand{\myvec}[1]{\ensuremath{\begin{pmatrix}#1\end{pmatrix}}}
\newcommand{\mydet}[1]{\ensuremath{\begin{vmatrix}#1\end{vmatrix}}}
%\numberwithin{equation}{section}
\numberwithin{equation}{subsection}
%\numberwithin{problem}{section}
%\numberwithin{definition}{section}
\makeatletter
\@addtoreset{figure}{problem}
\makeatother

\let\StandardTheFigure\thefigure
\let\vec\mathbf
%\renewcommand{\thefigure}{\theproblem.\arabic{figure}}
\renewcommand{\thefigure}{\theproblem}
%\setlist[enumerate,1]{before=\renewcommand\theequation{\theenumi.\arabic{equation}}
%\counterwithin{equation}{enumi}


%\renewcommand{\theequation}{\arabic{subsection}.\arabic{equation}}

\def\putbox#1#2#3{\makebox[0in][l]{\makebox[#1][l]{}\raisebox{\baselineskip}[0in][0in]{\raisebox{#2}[0in][0in]{#3}}}}
     \def\rightbox#1{\makebox[0in][r]{#1}}
     \def\centbox#1{\makebox[0in]{#1}}
     \def\topbox#1{\raisebox{-\baselineskip}[0in][0in]{#1}}
     \def\midbox#1{\raisebox{-0.5\baselineskip}[0in][0in]{#1}}

\vspace{3cm}


\title{Assignment 19}
\author{Jayati Dutta}





% make the title area
\maketitle

%\newpage

%\tableofcontents

\bigskip

\renewcommand{\thefigure}{\theenumi}
\renewcommand{\thetable}{\theenumi}
%\renewcommand{\theequation}{\theenumi}


\begin{abstract}
This is a simple document explaining how to express any matrix in square root form and how to express any matrix in Jordon form.
\end{abstract}

%Download all python codes 
%
%\begin{lstlisting}
%svn co https://github.com/JayatiD93/trunk/My_solution_design/codes
%\end{lstlisting}

Download all and latex-tikz codes from 
%
\begin{lstlisting}
svn co https://github.com/gadepall/school/trunk/ncert/geometry/figs
\end{lstlisting}
%


\section{Problem}
Use the result of Exercise 15 (that is, if $\vec{A}=\vec{I}+\frac{1}{2}\vec{N}-\frac{1}{8}\vec{N}^2$ then $\vec{A}^2 = \vec{I}+\vec{N}$) to prove that if $c$ is a non-zero complex number and $\vec{N}$ is a nilpotent complex matrix, then $(c\vec{I}+\vec{N})$ has a square root. Now use the Jordon form to prove that every non-singular complex $n \times n$ matrix has a square root.
 

\section{Solution} 
\begin{longtable}{|c|c|}
\hline
\multirow{3}{*}{} & \\
$\textbf{Given}$ & If $\vec{A}=\vec{I}+\frac{1}{2}\vec{N}-\frac{1}{8}\vec{N}^2$\\
& then $\vec{A}^2 = \vec{I}+\vec{N}$ that is, $\vec{I}+\vec{N}$ has a \\
& square root, where $\vec{N}$ is a nilpotent matrix.\\
& \\
\hline
\textbf{To prove} & \\
\hline
\textbf{1} & $(c\vec{I}+\vec{N})$ has a square root,that is $(c\vec{I}+\vec{N})= \vec{X}^2$\\
& where $c$ is a non-zero complex number, $\vec{N}$ is a \\
& nilpotent complex matrix and $\vec{X}$ is any matrix.\\
\hline
\textbf{2} & Every non-singular complex $n \times n$ matrix \\
& ($\vec{B}$) has a square root, that is, $\vec{B} = \vec{Y}^2$ where $\vec{Y}$\\
&  is any matrix.\\
\hline
\multirow{3}{*}{} & \\
\textbf{Proof 1} & As given in the problem statement,\\
& $\vec{A}^2 = \vec{I}+\vec{N}$ \\
& $\implies \vec{A}^2 = c\vec{I} - c\vec{I}+ \vec{I} +\vec{N}$\\
& $\implies \vec{A}^2 +c\vec{I}- \vec{I}= c\vec{I}+\vec{N}$\\
& $\implies \vec{A}^2 + (c-1)\vec{I}= c\vec{I}+\vec{N}$\\
& \\
\hline
\textbf{Conclusion} &\\
\hline
\multirow{3}{*}{} & \\
\textbf{Proof 2} & Let $\vec{B}$ is any non-singular complex matrix.\\
& So, $\vec{B}\vec{B}^{-1}= \vec{I}$\\
& As per the Jordon's Theorem, every square matrix B\\
&  is similar to a Jordon matrix $\vec{J}$, that is, $\vec{B}=\vec{P}\vec{J}\vec{P}^{-1}$\\
& Now, let consider a $n \times n$ nilpotent shift matrix $\vec{N}$\\
& $\vec{N}$ = \\
& $\myvec{0 & 1 &0 &...& 0\\0 & 0 & 1& ...& 0\\.&.&.&...&.\\.&.&.&...&.\\.&.&.&...&1\\0 & 0 &0 &...& 0}$\\
& $\vec{I}$ is a $n \times n$ identity matrix, so \\
& $c\vec{I}+\vec{N}$ = \\
& $\myvec{c & 1 &0 &...& 0\\0 & c & 1& ...& 0\\.&.&.&...&.\\.&.&.&...&.\\.&.&.&...&1\\0 & 0 &0 &...& c}$\\
& and this is a Jordon form.\\
& So, we can consider $\vec{J} = c\vec{I}+\vec{N}$ and \\
& as $c\vec{I}+\vec{N}= \vec{X}^2$ \\
& $\implies \vec{B}=\vec{P}\vec{X}^2\vec{P}^{-1}$\\
& $\implies \vec{B}=\vec{P}\vec{X}\vec{P}^{-1}\vec{P}\vec{X}\vec{P}^{-1}$\\
& $\implies \vec{B}=(\vec{P}\vec{X}\vec{P}^{-1})^2$\\
& $\implies \vec{B}=\vec{Y}^2$, where $\vec{Y} = \vec{P}\vec{X}\vec{P}^{-1}$\\
& This implies that $\vec{B}$ has a square root.\\
\hline
\textbf{Conclusion} & Hence it is proved that every non-singular complex \\
& $n \times n$ matrix has a square root.\\
& \\
\hline
\caption{$\textbf{Solution summary}$}
\label{table:1}
\end{longtable}
%\renewcommand{\theequation}{\theenumi}
%\begin{enumerate}[label=\thesection.\arabic*.,ref=\thesection.\theenumi]
%\numberwithin{equation}{enumi}
%\item Verification of the above problem using python code.\\
%\solution The  following Python code verifies the above solution.
%\begin{lstlisting}
%codes/multiplication_test.py
%\end{lstlisting}
%%%
%\end{enumerate}

\end{document}



